%%%%%%%%%%%%%%%%%%%%%%%%%%%%%%%%%%%%%%%%%%%%%%%%%
%% Einleitung %%
%%%%%%%%%%%%%%%%%%%%%%%%%%%%%%%%%%%%%%%%%%%%%%%%%

\chapter{Einleitung}
\label{ch:einleitung}	

\section{Motivation}
In den letzten Jahren erfährt die Automobilbranche einen massiven Umbruch.  Autonomes Fahren, E-Mobilität, Hybride, intelligente Sprachassistenzsysteme, personalisierte Apps, Infotainmentsysteme sind nur ein paar der rasant wachsenden Technologiefelder. Für den Fortschritt maßgebend ist die Arbeit der Forschungs- und Entwicklungsabteilungen. Sei es der Ausbau von Vakuum-Dünnschichtverfahren für Batteriesysteme oder die Verunreinigungsanalyse im Produktionsprozess; jede Komponente und jede Software, die in ein Auto gehört oder mit einem PKW zu tun haben könnte, sind Teil des Entwicklungsprozesses. Alleine in Deutschland wurden in 2017 circa 43 Milliarden Euro\footnote{Quelle: \href{https://de.statista.com/statistik/daten/studie/503258/umfrage/deutsche-automobilindustrie-ausgaben-fuer-forschung-und-entwicklung-weltweit}{statista.com} Stand: 17.06.2020.} für Forschung und Entwicklung im Automobilsektor investiert.

Der Klimawandel und die Endlichkeit der Öllagerstätten stellt eine technologische Herausforderung für die Automobilbranche dar. Neue Technologien benötigen Infrastrukturen sowie allgemein gültige Regeln ihrer Nutzung. Zusätzlich steigen die Anforderung an die einzelnen Komponenten des Autos. \blockquote{Teilprozesse und Elemente müssen in unterschiedliche Kontexten nicht nur für sich weitestgehend störungsfrei funktionieren. Sie wirken vielmehr aufeinander ein} \textcite[122]{berndt2014nicht}. Die Investitionsentscheidungen des Managements werden im Innovationskontext, mit der Diversität des Technologiefeldes und einer gleichzeitig immer umweltbewussteren Gesellschaft zunehmend risikobehaftet. Plug-In-Hybrid, Gasantrieb, Wasserstoff oder doch der saubere Diesel? Die Folgen dieser Entscheidungen können erst zukünftig bewertet und nur bedingt korrigiert werden. Ein bessere Kenntnis über das Innovationsverhalten von Konkurrenzfirmen, könnte diese Entscheidungen erleichtern. 

%In dieser Arbeit soll das Innovationsverhalten für den japanischen Automobilherstellers Honda und seine Konkurrenz, mit Hilfe eines Technologieraumes, dargestellt werden. 

\section{Der Technologieraum}

Eine Technologie ist ein komplexes, heterogenes Konglomerat aus verschiedenen Feldern, sie lässt sich durch verschiedenen Aspekte charakterisieren und wird neben der Wissenschaft bzw. dem Stand der Technik durch verschiedene Faktoren beeinflusst. Dynamischen Einflussfaktoren führen dazu, dass sich die Entwicklung von Technologien, ähnlich wie in Feldern der Mathematik oder Physik, chaotisch und scheinbar unvorhersehbar verhält \parencite{engelsman1994patent}. Dabei steht und der Erfolg von Unternehmen maßgeblich mit deren Innovationsverhalten. Studien der letzten Jahrzehnte haben gezeigt, dass die Produktivität eines Unternehmens oder einer Industrie, stark mit deren Ausgaben im Forschungs- und Entwicklungsbereich zusammenhängt \parencite{jaffe1993geographic}. 
 
Um zukunftsfähige Technologien in bestimmten Bereichen zu identifizieren, bietet es sich an das Innovationsverhalten von Unternehmen in den relevanten Technologiefeldern zu untersuchen. In diesem Rahmen eignet sich die Darstellung eines Technologieraumes. In eine technology space können mehrere Firmen, Technologien oder Technologiefelder in einem bestimmten Verhältnis zueinander dargestellt. In der Arbeit von \parencite{engelsman1994patent} wird der Technologieraum genutzt um zu zeigen, wie Firmen der CD/DVD- Industrie einen Innovationsschub durch Exploration anderer Industriebereichen gewinnen können. \textcite{almeida1997exploration} untersuchen Start-up-Unternehmen der Halbleiterindustrie in ihrer technologische Diversität aufgrund von Standortfaktoren. \textcite{alstott2017mapping} konstruieren ein Technologienetzwerk, mit dem der Zusammenhang zwischen einzelnen Technologiefelder dargestellt werden kann.
\textcite{zehtabchi2019measuring} analysiert Veränderungen des Technologieraums um Aussagen über die zeitliche Entwicklung einzelner Länder im Bereich des Autonomen Fahrens zu treffen   

Die ökonomische Zielrichtung des Technologieraums fällt in der Literatur oftmals unterschiedlich aus. Dennoch basieren unterschiedliche Darstellungsformen auf denselben Ansätzen\footnote{Im Rahmen einer strukturierten Literaturrecherche wurden jeweils die ersten fünf, nach Relevanz sortierten, wissenschaftliche Arbeiten für die Stichpunkte \glqq technological proxmity\grqq{} und  \glqq measuring firms technological position\grqq{} betrachtet. Des Weiteren wurden die Hauptarbeiten \textcite{stuart1996local} und \textcite{jaffe1986technological} auf relevante Zitationen untersucht.} So bieten meist Patentdaten die Basis für die Quantifizierung des technischen Knowhows einer Firma. Das Patent wird aufgrund einer technischen Erfindung vergeben, es bietet somit einen Innovationsindikator und ist zudem für Dritte zugänglich. Für den Technologieraum auf Patentbasis existieren zwei fundamentale Ansätze, die in der Literatur auf mehr oder weniger stark modifizierte Weise, Anwendung finden. 

In \textcite{jaffe1986technological} werden Innovationsindikatoren in verschiedenen Forschungsbereichen durch Firmenpatente in einzelnen Patentklassen abgebildet. Die Aktivität eines Unternehmens in einem bestimmten Forschungssektor wird demnach durch die Anzahl der Patente in der für diesen Sektor entsprechenden Patentklasse bestimmt.

Ein Patent stellt eine technische Erfindung dar, kann aber auch auf anderen Technologien basieren und diese zitieren. Zitieren also beispielsweise zwei Unternehmen die Patente ähnliche Technologien, werden sich die Firmen in einem Technologieraum dicht aneinander positionieren. Der auf Patentzitaten basierende Technologieraum wird erstmals in \textcite{stuart1996local} beschrieben.

Im Laufe der Arbeit sollen die beiden Ansätze genauer beschrieben und deren Zielrichtungen verdeutlicht werden. Des weiteren sollen die Modelle implementiert werden um den Technologieraum für das Unternehmen Honda und seine Konkurrenz darzustellen und zu interpretieren. Im letzten Teil der Arbeit wird eine weitere Methodik außerhalb der bestehenden Literatur vorgestellt, dabei werden Unternehmen mit Hilfe eines Themenmodells in einem Technologieraum dargestellt. Wir nutzen multidimensionale Skalierung um die technologischen Positionen der Firmen in einem zweidimensionalen Raum abzubilden. Diese Arbeit entstand in Kooperation mit der Firma Honda. Firmeninterne Technologieexperten konnten eine Validierung der Ergebnisse vornehmen und die Interpretation der entstandenen Technologieräume erleichtern.

Für die unterschiedlichen Ansätze wird eine gewisse Konstanz in den Ergebnissen sichtbar. Insgesamt bewegen sich die Automobilhersteller in den letzten Jahren, technologisch betrachtet, aufeinander zu. Die Firma Mazda stellt in diesem Kontext eine Ausnahme dar, so können wir sehen, dass sich das Unternehmen zunehmend von anderen Firmen distanziert. Grund dafür ist das setzten anderer technologischer Schwerpunkte. Mazda forscht in dem Bereich alternativer Motorentechnik (hybrider Wankelmotor). Wir können außerdem ein ähnliches Forschungsverhalten der japanischen Automobilherstellern Honda, Nissan und Toyota feststellen. So scheint das Unternehmen Toyota im Bereich der Elektromobilität ein führender Innovationsträger zu sein.

