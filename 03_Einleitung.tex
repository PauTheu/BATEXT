%%%%%%%%%%%%%%%%%%%%%%%%%%%%%%%%%%%%%%%%%%%%%%%%%
%% Einleitung %%
%%%%%%%%%%%%%%%%%%%%%%%%%%%%%%%%%%%%%%%%%%%%%%%%%

\chapter{Einleitung}
\label{ch:einleitung}	

\begin{itemize}
	\item Heranführung an Thema
	\item Technologieräume auf unterschiedliche Weise darstellen
	\item Hängt von Zielrichtung ab
	\item Ziel: Versuch makroökonomische Fragestellungen zu veranschaulichen und zu erklären
	\item Klassischer Ansatz: Fokus auf patentbasierte Ansätze
	\item Beispiel mit Experiment \parencite{engelsman1994patent}
	
\end{itemize}

Eine Technologie ist ein komplexes, heterogenes Kongolmerat aus verschiedenen Feldern, sie lässt sich durch verschiedenen Aspekte charakterisieren und wird neben der Wissenschaft bzw. dem Stand der Technik durch verschiedene Faktoren beeinflusst. Dynamischen Einflussfaktoren führen dazu, dass sich die Entwicklung von Technologien, ähnlich wie in Feldern der Mathematik oder Physik, chaotisch und scheinbar unvorhersehbar verhält.\parencite{engelsman1994patent}. Dabei steht und fällt aber der Erfolg von Unternehmen maßgeblich mit dem setzten auf die richtigen Technologien. Zahlreiche Studien der letzten Jahrzehnte haben gezeigt, dass die Produktivität eines Unternehmens oder einer Industrie, stark mit deren Ausgaben im Forschungs- und Entwicklungsbereich zusammenhängt.\parencite{jaffe1993geographic}. Um zukunftsfähige Technologien in bestimmten Bereichen zu identifizieren, bietet es sich also an zunächst das Innovationsverhalten von Unternehmen in diesen Bereichen zu untersuchen. 