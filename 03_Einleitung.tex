%%%%%%%%%%%%%%%%%%%%%%%%%%%%%%%%%%%%%%%%%%%%%%%%%
%% Einleitung %%
%%%%%%%%%%%%%%%%%%%%%%%%%%%%%%%%%%%%%%%%%%%%%%%%%

\chapter{Einleitung}
\label{ch:einleitung}	


Die Automobilbranche erfährt in den letzten Jahren einen gewaltigen Umbruch.  Autonomes Fahren, E-Mobilität, Hybride, intelligente Sprachassistenzsysteme, personalisierte Apps, Infotainmentsysteme sind nur ein paar der rasant wachsenden Technologiefelder. Das Auto erfindet sich gerade neu, maßgebend dafür ist die Arbeit in den Forschungs- und Entwicklungsabteilungen. Sei es der Ausbau von Vakuum-Dünnschichtverfahren für Batteriesysteme oder die Verunreinigungsanalyse im Produktionsprozess; jede Komponente und jede Software, die in ein Auto gehört oder mit einem PKW zu tun haben könnte, soll in irgendeiner Form ein Teil des Fortschritts sein. Alleine in Deutschland werden jedes Jahr circa 43 Milliarden Euro\footnote{Stand: 2017 URL: https://de.statista.com/statistik/daten/studie/503258/umfrage/deutsche-automobilindustrie-ausgaben-fuer-forschung-und-entwicklung-weltweit/ [17.04.2020].} für Forschung und Entwicklung investiert. Bei einem so diversen Technologiefeld und einem gleichzeitig so mächtigen Investitionsapparat wird die Frage nach den Entwicklungen von Technologien sowie dem Innovationsverhalten von Konkurrenzunternehmen zu einer zentralen Frage für jeden Automobilhersteller. Diese Arbeit soll sich mit dieser Problematik aus der Sicht des japanischen Automobilherstellers Honda auseinandersetzen.

Eine Technologie ist ein komplexes, heterogenes Konglomerat aus verschiedenen Feldern, sie lässt sich durch verschiedenen Aspekte charakterisieren und wird neben der Wissenschaft bzw. dem Stand der Technik durch verschiedene Faktoren beeinflusst. Dynamischen Einflussfaktoren führen dazu, dass sich die Entwicklung von Technologien, ähnlich wie in Feldern der Mathematik oder Physik, chaotisch und scheinbar unvorhersehbar verhält.\parencite{engelsman1994patent}. Dabei steht und fällt aber der Erfolg von Unternehmen maßgeblich mit dem setzten auf die richtigen Technologien. Studien der letzten Jahrzehnte haben gezeigt, dass die Produktivität eines Unternehmens oder einer Industrie, stark mit deren Ausgaben im Forschungs- und Entwicklungsbereich zusammenhängt.\parencite{jaffe1993geographic}. Um zukunftsfähige Technologien in bestimmten Bereichen zu identifizieren, bietet es sich also an zunächst das Innovationsverhalten von Unternehmen in diesen Bereichen zu untersuchen. In diesem Rahmen eignet sich die Formulierung eines \glqq technology space\grqq{}. Grundsätzlich können in einem Technologieraum mehrere Firmen, oder Technologien in einem bestimmten Verhältnis zueinander dargestellt werden. In der Literatur bieten Patentdaten meist die Basis für die Quantifizierung des technischen Knowhows. Ein Patent stellt eine technische Erfindung dar und ist für Dritte zugänglich. In \parencite{engelsman1994patent} werden Patentzitate genutzt um zu zeigen, wie Firmen der CD/DVD- Industrie einen Innovationsschub durch Exploration anderer Industriebereichen gewinnen können. \textcite{almeida1997exploration} untersuchen Start-up-Unternehmen der Halbleiterindustrie in ihrer technologische Diversität aufgrund von Standortfaktoren. \textcite{zehtabchi2019measuring} nutzt das Klassifikationssystem von Patenten um Aussagen über die zeitliche Entwicklung einzelner Länder im Bereich des Autonomen Fahrens zu treffen \textcite{alstott2017mapping} konstruiert ein Technologienetzwerk, mittels Analyse von Kozitationen in Patentklassen.  

Die ökonomische Zielrichtung des Technologieraums fällt in der Literatur oftmals unterschiedlich aus. Dennoch basieren unterschiedliche Darstellungsformen auf denselben Ansätzen. So bieten meist Patentdaten die Basis für die Quantifizierung des technischen Knowhows einer Firma. Das Patent wird aufgrund einer technischen Erfindung vergeben, es bietet also einen soliden Innovationsindikator und ist zudem für Dritte zugänglich. Für den Technologieraum auf Patentbasis existieren zwei fundamentale Ansätze, die in der Literatur auf mehr oder weniger stark modifizierte Weise, Anwendung finden. In \textcite{jaffe1986technological} werden Innovationsindikatoren in verschiedenen Forschungsbereichen durch Firmenpatente in einzelnen Patentklassen abgebildet. Die Aktivität eines Unternehmens in einem bestimmten Forschungssektor wird demnach durch die Anzahl der Patente in der für diesen Sektor entsprechenden Patentklasse bestimmt. Ein Patent stellt eine technische Erfindung dar, kann aber auch auf anderen Technologien basieren und diese zitieren. Zitieren also beispielsweise zwei Unternehmen die Patente gleicher Patentfamilien, liegt die Vermutung nahe, dass sie in einem Technologieraum eine ähnliche Position einnehmen werden. Der auf Patentzitaten basierende Technologieraum wird erstmals in \textcite{stuart1996local} beschrieben.

Im Laufe der Arbeit sollen die beiden Ansätze genauer beschrieben und deren Zielrichtungen verdeutlicht werden. Des weiteren sollen die Modelle implementiert werden um den Technologieraum für das Unternehmen Honda und seine Konkurrenz darzustellen und zu interpretieren. Im letzten Teil der Arbeit wird eine weitere Methodik außerhalb der bestehenden Literatur vorgestellt, dabei werden Unternehmen mit Hilfe eines Themenmodells in einem Technologieraum dargestellt. 

Für die unterschiedlichen Ansätze wird eine gewisse Konstanz in den Ergebnissen sichtbar. Insgesamt bewegen sich die Automobilhersteller in den letzten Jahren, technologisch betrachtet, aufeinander zu. Mazda stellt in diesem Kontext eine Ausnahme dar, so können wir sehen, dass sich das Unternehmen, zunehmend von anderen Firmen distanziert. Grund dafür ist das setzten anderer technologischer Schwerpunkte. Mazda forscht in dem Bereich alternativer Motorentechnik (Hybrider Wankelmotor). Wir können außerdem ein ähnliches Forschungsverhalten der japanischen Automobilherstellern Honda, Nissan und Toyota feststellen. So scheint das Unternehmen Toyota im Bereich der Elektromobilität ein führender Innovationsträger zu sein.