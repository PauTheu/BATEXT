\chapter{Der Technologieraum basierend auf Patentzitaten}
\label{ch:zitat}

In diesem Kapitel wird ein Technologieraum auf Basis von Patentzitaten vorgestellt. Zunächst werde ich die Paper \textcite{stuart1996local} und \textcite{kitahara2017technology} zusammenfassen und anschließend die darin vorgestellten Modelle implementieren.

\subsection{Grundlagen der Literatur}

\textcite{stuart1996local} stellen neben \textcite{jaffe1986technological} eine andere Möglichkeit vor, wie Unternehmen und deren Technologien in einem Technologieraum dargestellt werden können. Zunächst stellen die Autoren die \glqq lokale Suche\grqq{} als eine zentrale Voraussetzung für die Lokalisation von Firmen in einem Technologieraum fest. Bei der lokalen Suche geht es um die Annahme, dass Firmen ihre Forschungs- und Entwicklungskapazitäten in die Bereiche investieren, in denen sie bereits erfahren und profiliert sind. Somit ist es wahrscheinlicher, dass Daimler beispielsweise im nächsten Quartal ihre Forschungsgelder in die Weiterentwicklung des autonomen Fahrens investiert, als in Tierfutter. Ein zitatbasierter Ansatz setzt voraus, dass Unternehmen auf vergangenem Wissen aufbauen. Würden Firmen ihre Innovationsrichtungen also willkürlich wählen, wäre ein solche Herangehensweise sinnlos. Im Hauptteil ihrer Arbeit stellen die Autoren eine Methodik vor, wie sich - anhand von Überschneidungen in Patentzitaten - ein Maß für die Ähnlichkeiten verschiedener Firmen im Forschungs- und Entwicklungsbereich mathematisch als Distanzen berechnen lassen.

\textcite{kitahara2017technology} erweitern in ihrer Publikation den traditionellen Ansatz von \textcite{stuart1996local}, um Aussagen über die Entwicklung von Technologien in den Vereinigten Staaten zu treffen. Um präzisere Ergebnisse zu erzielen, berücksichtigen die Autoren dabei nicht nur \glqq first-order overlaps\grqq{} sondern auch \glqq second-order overlaps\grqq{}. Dieses Vorgehen ermöglicht es zusätzlich zu den Patentzitaten auf der ersten Ebene, die von den Patentzitaten auf der ersten Ebene wiederum zitierten Patente - die der zweiten Ebene - in das Modell aufzunehmen. 
	
	

\subsection{Zielsetzung der Studien}

Ziel der Arbeit von \textcite{stuart1996local} ist es ihren \glqq neu definierten\grqq{} Technologieraum durch ein eigenes Beispiel zu motivieren. Mithilfe des definierten Modells wir die Veränderungen der technologischen Positionen zehn japanischer Firmen in der Halbleiterindustrie über einen Zeitraum von 1982 bis 1992 in drei Zeitintervallen dargestellt. Die Darstellung erfolgt mittels multidimensionaler Skalierung. Änderung des strategischen und innovativen der Firmen Verhaltens sollen mittels der resultierenden Graphik begründet werden. So kann beispielsweise gezeigt werden, dass sich eine Firma aufgrund eines erfolgreichen Strategiewechsels im Technologieraum stark bewegt. Weiterhin versuchen die Autoren Gruppierungen von Firmen im Technologieraum zu interpretieren um möglicherweise die Formierung strategischer Allianzen zu begründen. \\

In \textcite{kitahara2017technology} wird ein allgemeineres Bild betrachtet. Anstatt den Effekt der technologischer Distanz auf Firmenebene zu betrachten, untersuchen die Autoren die Beziehung zwischen innovativem Output und Technologieraumformationen. Es wird versucht folgende Kernfrage zu beantworten: Welche Art von Verteilung stimuliert Innovation? Konstruiert man ein Beispiel wird die Fragestellung klarer. Wir nehmen an, alle Firmen in dem Technologieraum sind auf eine technologische Position konzentriert. Wie \textcite{jaffe1986technological} suggeriert, werden sich alle Unternehmen aufgrund von Spillovers unfreiwillig gegenseitig helfen, denn sind sich alle Unternehmen ähnlich ist es einer Firma schnell möglich innovative Technologien der anderen Firmen selbst umzusetzen. Auf der anderen Seite sind Übertragungseffekte aber gering, wenn sich die Firmen über den kompletten Technologieraum verteilen. \textcite{kitahara2017technology} untersuchen die Mitte dieser beiden Extrema. Angenommen es gibt zwei, voneinander entfernte Pole, um die sich die Firmen gruppieren. Die beiden Firmengruppen nutzen jeweils unterschiedliche Technologien und wetteifern darum, welche Technologie zum Marktstandard wird.


\subsection{Der Technologieraum nach \textcite{stuart1996local}}

Nimmt man an, dass Unternehmen ihre Forschungsrichtungen \glqq lokal suchen\grqq{}, ist die patentbasierte Definition eines Technologieraums relativ intuitiv. Haben Firmen ähnliche Wissensquellen werden sie ähnliche Technologien besitzen und in ähnlichen Technologiesektoren forschen. Demnach werden sich die Firmen im Technologieraum nahe stehen. Die technologische Distanz ist nach \textcite{stuart1996local} also ein Maß nach Gemeinsamkeiten in Wissensbasen. \\

Der Koeffizient $\alpha_{ij}$ repräsentiert das Verhältnis, der Anzahl an Wissensquellen auf der Firma j aufbaut, die auch von Firma i genutzt werden. Werden Patentzitate als Analogien für Wissensquellen behandelt so ist der Wert $\alpha_{ij}$ der Anteil der zitierten Patente von Firma j, die auch Firma i zitiert. Die \glqq competition-coefficients\grqq{} sind Werte zwischen 0 und 1, wobei die Firmen bei $\alpha_{ij}= 0$ auf keine gemeinsamen Patente zitieren. Für ein System von N Firmen, kann eine asymmetrische $NxN$ Matrix aufgestellt werden. Die Einträge dieser \glqq community matrix\grqq{} sind die Koeffizienten $\alpha_{ij}$ und $\alpha_{ji}$ für $(j = 1,2, ..., N; i = 1, 2, ..., N; i \neq j)$.


\begin{figure}[!ht]
\centering
\includegraphics[width=0.8\linewidth]{files/bspzit.PNG}
\caption{Beispiel: drei Firmen zitieren Patente}
\label{bspzit}
\end{figure}

Die Abbildung \ref{bspzit} zeigt drei Firmen A, B und C und die Patente 1 bis 9. Der Pfeil von einer Firma zu einem Patent impliziert ein Zitat. So zitiert beispielsweise Firma A insgesamt vier Patente (1, 2, 3 und 5). Firma B zitiert lediglich zwei Patente (3 und 5). Der Koeffizient $\alpha_{AB}$, ist der Anteil der zitierten Patente von Firma B, die auch von Firma A zitiert werden. Es ergibt sich $\alpha_{AB} = \frac{2}{4} = 0.5$. Da Firma B alle Patente zitiert, die auch von Firma A zitiert werden ergibt sich umgekehrt $\alpha_{BA} = 1$. Die Werte entlang der Hauptdiagonale liefern keine Informationen und werden mit 0 besetzt. Für die Firmen A, B und C ergibt sich folgende \glqq community matrix\grqq{}:

\vphantom{dasdddddddddddddddddddddddddddddddddddddddddddddddddddddddddddddddddddddddddddddddddddddddddddddddddddddddddddddddddddd}

$\begin{pNiceMatrix}[first-col, last-row]% don't forget the %
\text{A } \; \; \;    & 0 & 0.50 & 0.75   \\
\text{B   } \; \; \;   & 1 & 0 & 1  \\
\text{C   }  \; \; \; & 0.375 &  0.25 & 0  \\
                          &\text{A} & \text{B} & \text{C} 
                      
\end{pNiceMatrix}$

\vphantom{dasdddddddddddddddddddddddddddddddddddddddddddddddddddddddddddddddddddddddddddddddddddddddddddddddddddddddddddddddddddd}

Die \glqq community matrix\grqq{}, dient als Ausgangspunkt. Firma A baut auf 100\% des Wissens der Firma B und 37.5\% der Firma C auf. Der Spaltenvektor einer Firma lässt sich auch als deren \glqq technologische Nische\grqq{} verstehen. Um zu untersuchen, wie sich die Firmen im Technologieraum über die Zeit bewegen, definieren die Autoren zunächst die Distanz zwischen Firma i und j zum festen Zeitpunkt $t_m$. Die (euklidische) Distanz zwischen Firmen ist ein Maß davon, inwieweit sich die technologischen Nischen der Firmen i und j mit denen aller anderen Firmen k überschneiden.

\begin{equation}
\label{sequ1}
d_{jit_m} \equiv d_{ijt_m} = \bigg\{ \sum_{k=1}^{n} [(\alpha_{ikt_m} - \alpha_{jkt_m})^2 + (\alpha_{kit_m} - \alpha_{kjt_m})^2]\bigg\}^\frac{1}{2}, k \neq i, j
\end{equation}

Die $\alpha$'s sind die Wettbewerbskoeffizienten der i'ten und j'ten Dyaden zum Zeitpunkt $t_m$. Der erste Term $(\alpha_{kit_m} - \alpha_{kjt_m})$ beschreibt die Differenz Nischenüberschneidungen zwischen Firma i und anderen Firmen k und der Nischenüberschneidungen von Firma j und anderen Firmen k. Der zweite Term $(\alpha_{ikt_m} - \alpha_{kjt_m})$ ist die Gegenrichtung. Er gibt an inwieweit sich die technologischen Nischen der Firma k mit denen der Firmen i und j überschneiden. \\

Um die Distanz zwischen Firmen zu zwei verschiedenen Zeitpunkten festzuhalten, wird weiterhin eine Matrix nach folgender Metrik definiert. 

\begin{equation}
\label{sequ2}
d_{it_ljt_m} \equiv d_{jt_mit_l} = \bigg\{ \sum_{k=1}^{n} [(\alpha_{ikt_l} - \alpha_{jkt_m})^2 + (\alpha_{kit_l} - \alpha_{k_jt_m})^2]\bigg\}^\frac{1}{2}, k \neq i, j
\end{equation}


Die Idee ist Analog zu \ref{sequ1}, nur werden jetzt die Differenzen der Nischenüberschneidungen über die Zeitpunkte $t_l$ und $t_m$ gebildet. Sind alle Firmen in jeder Zeitperiode aktiv ergeben sich für die symmetrische Distanzmatrix D, $N*T$ Zeilen und $N*T$ Spalten, wobei $N$ die Anzahl der Firmen, und $T$ die Anzahl der Zeitperioden ist. Im letzten Schritt wird die Matrix mittels multidimensionaler Skalierung (siehe Abschnitt \ref{mds}) graphisch dargestellt.

\subsection{Die Firma Mitsubishi im Technologieraum}

\textcite{stuart1996local} wenden die oben beschrieben Methodik an um Positionsveränderungen zehn japanischen Firmen in der Halbleiterindustrie darzustellen. Für den Technologieraum wurden die Jahre 1982, 1987 und 1992 gewählt. Da es zehn Firmen zu je drei Zeiträumen gibt, hat der resultierende Graph 30 Punkte. \textcite{stuart1996local} teilen den Technologieraum zunächst in zwei Gruppen ein. Auf der einen Seite befinden sich Firmen, die \glqq einfache\grqq{}, auf den Mainstream-Konsum basierte Technologien produzieren. Auf der anderen Seite befinden sich Unternehmen die sich in komplexeren, industriellen Technologiesektoren aufhalten. So produzieren diese Firmen die Kerntechnologien die in einem jeden Computer stecken (beispielsweise Schaltungslogik für Prozessoren). Das Unternehmen Mitsubishi, beschloss Mitte der 80er Jahre aufgrund von Prognosen einen strategischen Wechsel von dem Konsummarkt in den Industriemarkt vorzunehmen. Maßnahmen waren dabei unter anderem eine drastische Erhöhung der Forschungs- und Entwicklungsausgaben um in den Markt für Speicherentwicklung vorzudringen, oder das Herstellen von Intel's Mikroprozessoren. Mithilfe ihrer Darstellung (siehe \ref{mitsu}) können die Autoren zeigen, dass dieser Strategiewechsel erfolgreich war. Mitsubishi bewegt sich innerhalb des betrachteten Zeitraums im Technologieraum weiter weg von Firmen wie Sony und immer näher zu innovativen Unternehmen wie Hitachi und Toshiba. 

\begin{figure}[!ht]
\centering
\includegraphics[width=1.0\linewidth]{files/mitsu.PNG}
\caption{Technologische Positionen von Firmen in der japanischen Halbleiterindustrie \parencite[S. 30]{stuart1996local}}
\label{mitsu}
\end{figure}

\subsection{Strategische Allianzen}

Im Gegensatz zu Spillovers hat eine Firmenallianz symbiotische Eigenschaften. Austausch von Wissen und Technik, soll einen positiven Einfluss auf die Marktposition aller beteiligte Firmen haben. \textcite{stuart1996local} fanden im Rahmen ihrer Recherche 35 Fälle von Wissensaustausch zwischen den zehn Firmen. \\

Wie oben bereits erwähnt teilen die Autoren ihren Technologieraum in zwei Bereiche ein. Dabei werden die Firmen der linken Technologieraumhälfte als \glqq core\grqq{} bezeichnet. Diese Firmen sind besonders innovativ und können hohe Marktanteile aufweisen. Dazu gehören unter anderem die Firmen Hitachi, Toshiba, NEC, Fujitsu und Mitsubishi nach ihrer Neuorientierung. Auf der anderen Seite des Technologieraums befinden sich Firmen der \glqq periphery\grqq{}. Firmen wie Sony und Sanyo, die sich auf den Konsumhandel konzentrieren. \\

Firmen im \glqq core\grqq{}-Bereich des Technologieraums sind nach \textcite{stuart1996local} an insgesamt 31 der 35 Partnerschaften beteiligt. Strategische Allianzen finden sich also fast ausschließlich in die Richtungen \glqq core-to-core\grqq{}, beziehungsweise \glqq core-to-periphery\grqq{}. Mitsubishi, die Firma die sich innerhalb des Technologieraumes am extremsten bewegt hat, war in allen drei Jahren an den meisten Firmenpartnerschaften beteiligt. Im allgemeinen korreliert die Bewegungsdistanz einer Firma im Technologieraum positiv mit der Anzahl ihrer eingegangenen Partnerschaften \parencite[S. 34]{stuart1996local}. Gerade das Wechseln des Technologieclusters ist für Firmen nur in seltenen Fällen mit eigener Innovationskraft zu bewältigen. Nach \textcite{stuart1996local} besteht damit ein klarer Zusammenhang zwischen der Affinität einer Firma strategischen Allianzen einzugehen und ihrer Positionierung im Technologieraum. Auch wir konnten im vorherigen Kapitel eine Annäherung der Firma Ford und der Firma Toyota feststellen. Diese Bewegung könnte mitunter durch eine, im Jahre 2011 eingegangene Partnerschaft im Hybridsektor\footnote{Quelle: \hyperlink{https://www.forbes.com/sites/altheachang/2011/08/31/ford-toyota-hybrid-partnership/}{Forbes} Stand: 09.10.2020} begründet sein.


%\begin{itemize}
%
%\item firmen in der halbleiterindustrie darzustellen
%\item etc.
%\item Zitatbasierter Ansatz
%\item Zitatgrundlagen erklären - Wissensflüsse (Paper mit Zitattest)
%
%\end{itemize}



	
\subsection{Erweiterung des Ansatzes nach \textcite{kitahara2017technology}}	

\textcite{kitahara2017technology} erweitern den von \textcite{stuart1996local} entwickelten, auf Zitaten basierenden Ansatz, um eine weitere Ebene. Die Intuition dahinter ist, dass Zitate von Zitaten (Zitate der „zweiten Ebene“) für die Wissensflüsse weitere, wichtige Informationen beinhalten können. In diesem Abschnitt soll das Modell anhand eines eigenen Beispiels veranschaulicht werden. 


\begin{figure}[!ht]
\centering
\includegraphics[width=0.8\linewidth]{files/kitbsp.PNG}
\caption{Patentzitate der Firma Blau (links) und Firma Rot}
\label{kitbsp}
\end{figure}

Die Abbildung \ref{kitbsp} zeigt zwei fiktive Firmen (Rot und Blau)\footnote{Weiter: Firma R und Firma B} und ihre jeweiligen Patentzitate. So zitiert Firma R Die roten Patente 1, 2, 3 und 5 unterschiedlich oft. Analog zitiert Firma B die blauen Patente 1, 2, 4, 7 und 8. Die Patentzitate der zweiten Ebene werden durch Pfeile impliziert. So zitiert beispielsweise Patent 3, wiederum die Patente 8 und 9. \textcite{kitahara2017technology} führen vier Gleichungen ein, die alle Zitationsbeziehungen einfangen. 

\begin{equation}
\label{kitequ1}
\omega_{ij}^1 \equiv |O(P_i, P_j)| + |O(P_j, P_i)|
\end{equation}

Sei $|P|$ die Anzahl der Patente in einer Liste $P$. Seien weiterhin $O(P_i, P_j)$, alle Patente in $P_i$, die sich mit den Patente in $P_j$ überschneiden (Wiederholungen inklusive). Wir erhalten für $P_R = \{1, 1, 1, 2, 2, 4, 4\}$ und $P_B = \{1, 1, 2, 4, 4, 4\}$, $|P_R| = 7$ und $|P_B| = 6$ . $\omega_{ij}^1$ zählt die Patente im Schnittbereich der beiden Firmen. Nach Gleichung \ref{kitequ1} erhalten wir für unsere Firmen: $\omega_{RB}^1 = \omega_{BR}^1 = 13$.  Im Unterschied zu \textcite{stuart1996local}, werden auch Mehrfachzitate in die Berechnung aufgenommen. Auch hier ist die Intuition klar. Zitiert eine Firma dasselbe Patent mehrfach, so wird es für sie eine bedeutendere technologische Rolle spielen, als ein einfach zitiertes Patent. \\

Auch wenn sich Zitate nicht direkt überschneiden, könnten die Patente indirekt technologisch verknüpft sein. Seien also $\tilde{P}_{ij}$\footnote{Man beachte: $\tilde{P}_{ij}$ kann auch Wiederholungen beinhalten}, die Patente aus $P_i$, die sich nicht mit $P_j$ überschneiden (Patente außerhalb der Schnittmenge). Weiterhin seien $C_i(p)$, Patent $p$'s Zitate, mit $p \in \tilde{P}_{ij}$ . Auch hier berücksichtigen wir also nur die Patent(zitate), die noch nicht durch die Gleichung \ref{kitequ1} abgedeckt wurden. Den Elementen aus $\tilde{P}_{ij}$ wird ein positives Gewicht $k$ zugeordnet, wenn $p_k$ ein Patent in $\tilde{P}_{ij}$ zitiert oder von einem Patent in $\tilde{P}_{ji}$ zitiert wird.  Für die erste Komponente der \glqq second-order overlaps\grqq{} gilt:

\begin{equation}
\label{kitequ2}
\omega_{ij}^{21} = \sum_{k=1}^{n_{ij}} \frac{|O(C_i(p_k), \tilde{P}_{ij})|}{|(C_i(p_k)|}
\end{equation}

Wir erhalten für $\omega_{RB}^{21} = \frac{\{|8|\}}{1} + \frac{\{|8|\}}{2} + \frac{\{|8|\}}{2} + \frac{\{|8|\}}{2} = 1 + 0.5+  0.5 + 0.5 = 2.5$, veranschaulicht in Abbildung \ref{kitbsp2}. $\omega_{ji}^{21}$ wird analog zur Gleichung \ref{kitequ2} definiert. Für unser Beispiel ist $\omega_{BR}^{21}= 1$. \\

\begin{figure}[!ht]
\centering
\includegraphics[width=0.8\linewidth]{files/kitbsp2.PNG}
\caption{Patentzitate der zweiten Ebene, $\omega_{RB}^{21}$}
\label{kitbsp2}
\end{figure}

Mit der zweiten Komponente der \glqq second-order overlaps\grqq{}, werden alle \glqq übrigen\grqq{} Patente verrechnet. Seien $\tilde{P}^{\prime}_{ij}$ also die Patente in $\tilde{P}_{ij}$, die nicht mit den Patente in $\tilde{P}_{ji}$ überlappen und $\tilde{n}^{\prime}_{ij}$ die Anzahl der Patente in $\tilde{P}^{\prime}_{ij}$. Es gilt:
 
\begin{equation}
\label{kitequ3}
\omega_{ij}^{22} = \sum_{k=1}^{n^{\prime}_{ij}} \frac{|O(C_i(p_k), C_j(\tilde{P}_{ji}))|}{|C_i(p_k)|}
\end{equation}

$\omega_{ji}^{22}$ ist wieder analog definiert. Da die Menge $\tilde{P}^{\prime}_{RB}$ leer ist erhalten wir für $\omega_{RB}^{22} = 0$. Für die Gegenrichtung (Abbildung \ref{kitbsp3}) erhalten wir $\tilde{P}^{\prime}_{BR} =  \frac{|O(C_B(7), \{9\})|}{1} + \frac{|O(C_B(7), \{9\})|}{1}=  \frac{\{|9|\}}{1} + \frac{\{|9|\}}{1}= 2$. 

\begin{figure}[!ht]
\centering
\includegraphics[width=0.8\linewidth]{files/kitbsp3.PNG}
\caption{Patentzitate der zweiten Ebene, $\omega_{BR}^{22}$}
\label{kitbsp3}
\end{figure}








