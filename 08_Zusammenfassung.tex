\chapter{Fazit}
\label{ch:zsm}

\section{Zusammenfassung}

Wir haben gesehen mit welchen Methoden sich ein Technologieraum darstellen lässt, welchen Zweck eine solche Darstellung erfüllt und wie der Technologieraum für das Unternehmen Honda und seine Konkurrenz aussehen kann. Im Laufe der Arbeit haben wir uns den Patentdaten angenähert. So wurde die innovative Aktivität zunächst als eine Verteilung von Patenten in verschiedenen Patentklassen abgebildet. Im nächsten Schritt haben wir Überschneidungen technologischer Nischen mittels Patentzitaten untersucht. Als letztes haben wir die Texte von Patentauszügen betrachtet und dabei ein Themenmodell mit dem ersten Ansatz kombiniert. 

Der Technologieraum kann sich zwischen den verschiedenen Ansätzen unterscheiden, die wahre Verteilung der Firmen wird dabei vermutlich ein Mittelmaß sein. Die Interpretation der Technologieräume ist im Allgemeinen für Außenstehende nicht sonderlich intuitiv und sollte immer einer gewissen Skepsis unterliegen. Allerdings lassen sich die Ansätze an sich als relativ Robust beobachten. Wiederkehrende Muster in den einzelnen Technologieräumen lassen uns zudem einige gröbere Aussagen treffen. Wir erkennen dass die Klimaproblematik, einige etablierte Automobilunternehmen dazu zwingt im Forschungskontext zusammenzurücken. Andere Unternehmen wie die Firma Mazda scheinen riskantere technologische Schwerpunkte zu treffen und sich von dem Gesamtkonsens zu entfernen. Wir erkennen, dass die japanische Automobilhersteller, darunter Toyota, Honda und Nissan ähnliche Innovationsstrategien verfolgen. Bedingt durch hohe Patentzahlen kann man davon ausgehen, dass dabei die Firma Toyota den Ton angibt. Insbesondere im Bereich der Elektromobilität haben die Japaner im Vergleich zu westlichen Automobilherstellern die Nase vorne. \glqq Externe Faktoren wie wachsendes Umweltbewusstsein, strengere Emissions- und Verbrauchsnormen, gestiegene Benzin-Preise und öffentliche Förderung für den Kauf von Hybridfahrzeugen spielen dabei zweifelsohne eine Rolle\grqq{} \parencite[124]{berndt2014nicht}. Der Erfolg von Automobilunternehmen ist gerade für den Wirtschaftsstandort Deutschland essentiell, dabei wird eine so explosionsartige Expansion im Bereich der alternativen Energieträger, ohne eine starke wirtschaftliche Belastung kaum möglich sein. Nichtsdestotrotz erkennen wir wie westliche Unternehmen wie Bosch, Ford und GM Innovation in diesem Sektor vorantreibt und sich den japanischen Unternehmen annähern. 




\section{Ausblick}


Mit Hilfe eines Themenmodells  lassen sich komplexe Technologien nach allgemein verständlichen Begriffen ordnen. Somit ist auch der Leihe im Stande einen schnellen Einblick in die Daten zu gewinnen. Allerdings ist die Aussagekraft unseres Themenmodells beschränkt. Wir finden einige, gewichtige Auffangthemen, in denen die Themen nicht sonderlich intuitiv, beziehungsweise sinnvoll erscheinen. Um das Modell zu optimieren sollte die Liste der stopwords, um weitere, für Automobiltechnologien spezifische, Begriffe erweitert werden. Auch die Modellparameter könnten weiter angepasst werden. Dabei sollten explorativ verschiedene Modelle in ihrer Qualität verglichen werden. Um den nationalen bias zu minimieren sollten außerdem die Patentanmeldung überprüft werden. Dabei könnten im Kontext der Datenaufbereitung beispielsweise Patente gefiltert werden, die lediglich in einem Patentamt angemeldet wurden. In unseren Darstellungen betrachten wir einen Technologieraum auf Firmenebene. In der Literatur finden wir verschiedene Zielrichtung, so könnte auch unser Technologieraum erweitert werden. Beispielsweise könnte auch das Innovationsverhalten einzelner Länder untersucht werden.