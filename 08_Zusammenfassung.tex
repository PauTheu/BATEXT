\chapter{Zusammenfassung}
\label{ch:zsm}

Der Klimawandel und die Endlichkeit der Öllagerstätten stellt eine technologische Herausforderung für die gesamte Automobilbranche dar. Neue Technologien benötigen Infrastrukturen sowie allgemein gültige Regeln ihrer Nutzung. Zusätzlich steigen die Anforderung an die einzelnen Komponenten des Autos. \glqq Teilprozesse und Elemente müssen in unterschiedliche Kontexten nicht nur für sich weitestgehend störungsfrei funktionieren. Sie wirken vielmehr aufeinander ein\grqq{} \parencite[122]{berndt2014nicht}. Die Investitionsentscheidungen des Managements im Entwicklungskontext werden mit steigender Marktvolatilität zunehmend risikobehaftet. Die Folgen dieser Entscheidungen können erst zukünftig bewertet und nur bedingt korrigiert werden. Ein bessere Kenntnis über das Innovationsverhalten von Konkurrenzfirmen, könnte diese Entscheidungen erleichtern. Dazu können beispielsweise alle relevanten Firmen aufgrund ihres Forschungsverhaltens in ein Verhältnis zueinander gesetzt werden. In dieser Arbeit wurde gezeigt wie sich ein solcher Technologieraum auf verschieden Weise darstellen lässt. Basis für die Analyse bieten Patentdaten. Dabei haben wir uns im Verlauf der Arbeit den Daten \glqq angenähert\grqq{}. So wurde die innovative Aktivität zunächst als eine Verteilung von Patenten in verschiedenen Patentklassen abgebildet. Im nächsten Schritt haben wir Überschneidungen technologischer Nischen mittels Patentzitaten untersucht. Als letztes haben wir die Texte von Patentauszügen betrachtet und dabei ein Themenmodell mit dem ersten Ansatz kombiniert. 

Die Aussagekraft des Technologieraums kann sich je nach Ansatz unterscheiden, die wahre Verteilung der Firmen wird sich letztlich als Durchschnitt der verschiedenen Ansätze ergeben. Die Interpretation der Technologieräume ist im Allgemeinen für Außenstehende nicht sonderlich intuitiv und sollte immer einer gewissen Skepsis unterliegen. Allerdings lassen sich die Ansätze an sich als relativ Robust beobachten. Wiederkehrende Muster in den einzelnen Technologieräumen lassen uns zudem einige gröbere Aussagen treffen. Wir erkennen dass die Klimaproblematik, einige etablierte Automobilunternehmen dazu zwingt im Forschungskontext zusammenzurücken. Andere Unternehmen wie die Firma Mazda scheinen riskantere technologische Schwerpunkte zu treffen und sich von dem \glqq Gesamtkonsens \grqq{} zu entfernen. Wir erkennen, dass die japanische Automobilhersteller, darunter Toyota, Honda und Nissan ähnliche Innovationsstrategien verfolgen. Bedingt durch hohe Patentzahlen kann man davon ausgehen, dass dabei die Firma Toyota den Ton angibt. Insbesondere im Bereich der Elektromobilität haben die Japaner im Vergleich zu westlichen Automobilherstellern die Nase vorne.  \glqq Externe Faktoren wie wachsendes Umweltbewusstsein, strengere Emissions- und Verbrauchsnormen, gestiegene Benzin-Preise und öffentliche Förderung für den Kauf von Hybridfahrzeugen spielen dabei zweifelsohne eine Rolle\grqq{} \parencite[124]{berndt2014nicht}. Der Erfolg von Automobilunternehmen ist gerade für den Wirtschaftsstandort Deutschland essentiell, dabei wird eine so explosionsartige Expansion im Bereich der alternativen Energieträger, ohne eine starke wirtschaftliche Belastung kaum möglich sein. Nichtsdestotrotz erkennen wir wie westliche Unternehmen wie Bosch und GM Innovation in diesem Sektor vorantreibt und sich den japanischen Unternehmen annähern. 
