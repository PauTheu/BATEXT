\chapter{Kitahara}
\label{ch:kitahara}

\subsection{Grundlagen}
\begin{itemize}

\subsubsection{Erweiterung des zitatbasierten Ansatzes}

\item 2nd order overlaps etc
\item Zitatgrundlagen erklären - Wissensflüsse (Paper mit Zitattest)


\end{itemize}	


	
\subsubsection{Zielsetzung}	

\begin{itemize}

\item shift in amerikanischer industrie erklären (polarisation)
\item etc.
	
\end{itemize}
	
\subsubsection{Ergebnis}	

\begin{itemize}

\item 
	
\end{itemize}	
	


\subsection{Umsetzung}

\subsubsection{Zitate auf der zweiten Ebene}

\begin{itemize}

\item Beispiel hier (mit PP erstellt)

\end{itemize}
\subsubsection{Durchführung}

\begin{itemize}

\item ergänzung der datenbasis um zweite zitatebene
\item Berechnung der omegawerte erklären
\item etwas auf code eingehen (bspw. Laufzeitprobleme)
\item einteilung in mengen (sets) und typen
\item vielzitierte Patente

\end{itemize}

\subsubsection{MDS}

\subsubsection{Ergebnisse}

\end{itemize}