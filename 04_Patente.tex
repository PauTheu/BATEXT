%%%%%%%%%%%%%%%%%%%%%%%%%%%%%%%%%%%%%%%%%%%%%%%%%
%% Einleitung %%
%%%%%%%%%%%%%%%%%%%%%%%%%%%%%%%%%%%%%%%%%%%%%%%%%

\chapter{Patente}
\label{ch:Patente}	% This allows later to reference the chapter 


\begin{itemize}

           \item in diesem Abschnitt Funktion von Patenten erklären, da diese Basis für weitere Evaluationen darstellt
	\item Ein Patent gewährt dem Anmelder ein Besitz- und Verfügungsrecht für die kommerzielle Nutzung einer Erfindung
	\item Erfindung muss neu und nichttrivial sein (Def. aus PatG nutzen); Stand der Technik erläutern
	\item PatG: was keine Erfindung darstellt
	\item Wird ein Patent gewährt, wird ein für die Öffentlichkeit einsehbares Dokument erstellt.
	\item Dokument beinhaltet eine Reihe von Informationen: Erfinder, Arbeitgeber, etc., hier wichtig: Referenzen bzw. Zitationen
	\item Besitzen eine rechtliche Funktion: Begrenzen den Umfang der Verfügungsrechte ausgehend von dem Patent
	\item Sicherstellen, dass das gewährte Patent einen nichttrivialen Wissensbeitrag im Vergleich zu dessen Vorgängern erreicht.
	\item Im Kontext hier bedeutet: Wenn Patent Y Patent X zitiert, wird Patent Y auf dem Wissen von Patent X aufbauen.
	\item Kritik nach  \parencite{karki1997patent}
	\item propensity der Patentklassen
\end{itemize}

%% Weitere Untergliederungen:
\subsection{Grundlegendes}

In diesem Abschnitt soll die grundlegende Funktion von Patenten erklärt werden, da diese unsere Datenbasis sein werden und damit die Grundlage für alle kommenden Evaluationen darstellen. Ein Patent gewährt dem Anmelder ein Besitz- und Verfügungsrecht für die kommerzielle Nutzung einer Erfindung. 


\subsubsection{Klassifizierung}

\begin{itemize}

\item ipc und cpc

\end{itemize}

\subsubsection{Datenbasis}

\begin{itemize}

\item panda dataframes
\item plots

\end{itemize}