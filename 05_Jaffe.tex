%%%%%%%%%%%%%%%%%%%%%%%%%%%%%%%%%%%%%%%%%%%%%%%%%
%% Kapitel 1 %%
%%%%%%%%%%%%%%%%%%%%%%%%%%%%%%%%%%%%%%%%%%%%%%%%%

\chapter{Jaffe}
\label{ch:jaffe}

Im ersten Teil dieses Kapitels werden die Papiere \parencite{jaffe1986technological} und \parencite{jaffe1989characterizing} zusammengefasst. Im zweiten Teil soll ein Technologieraum, nach der Definition des Autors, für die Firma Honda und ihre Konkurrenz nachgebildet und analysiert werden.


\section{Literatur}
\subsection{Grundlagen}

	

Mit seinem Papier \parencite{jaffe1986technological} bietet Adam B. Jaffe einen wichtigen Grundbaustein für die Quantifizierung von Forschungs- und Entwicklungsarbeiten von Unternehmen. Offizielle innovatorische Kennzahlen von Unternehmen, wie z.B. Forschungskosten sind eindimensional und wenig aussagekräftig. Um die technologischen Position von Firmen zu charakterisieren bedient sich \parencite{jaffe1986technological} eines patentbasierten Ansatz. Jedes Unternehmen forscht in verschiedenen Sektoren. Alle relevanten Wirtschaftszweige werden mithilfe des Patentklassifikationssystems von 328 Patentklassen\footnote{dabei handelt es sich um das Klassifikationssystem des National Bureau of Economic Research (NBER)} in insgesamt 21 Clustern zusammengefasst. Es werden 260.000 Patente von 1700 Firmen über einen Zeitraum von zehn Jahren (1969 - 1979) betrachtet. \textcite{jaffe1986technological} kann zeigen, dass die Produktivität von Firmen in F\&E positiv von der innovatorischen Arbeit ihrer \glqq technologischen Nachbarn\grqq{} beeinflusst wird und Firmen ihr Innovationsverhalten ändern wenn sie eine Gelegenheit dazu bekommen. \parencite{jaffe1989characterizing} baut inhaltlich auf \parencite{jaffe1986technological}, hier werden zunächst zehn Firmen aus unterschiedlichen Sektoren in einem Technologieraum abgebildet. Dabei setzt der Autor seinen Fokus sowohl auf die technologischen Positionen der Unternehmen und die mathematischen Hintergründe des Technologieraums, als auch auf das Bilden der (industriellen) Cluster ausgehend von den Patentklassen. In \textcite{jaffe1986technological}, werden primär die statistischen Grundlagen für die Untersuchung von Wissensübertragungseffekte (\glqq Spillovers\grqq{}), bereitgestellt.


\subsection{Zielsetzung}	

Private Firmen investieren Ressourcen in Forschung und Entwicklung um wirtschaftlich nützliches Wissen zu \glqq generieren\glqq. Der Erfolg dieser Investitionen variiert stark, teilweise durch nicht messbare Zufallsfaktoren und teilweise durch systematische Effekte, ausgelöst durch das wirtschaftliche und technologische Umfeld einer Firma. Es sind diese Effekte (\glqq Spillovers\grqq{}) die in \parencite{jaffe1986technological} untersucht werden.
Die primäre makroökonomische Zielsetzung in \parencite{jaffe1986technological} und \parencite{jaffe1989characterizing} ist das konkretisieren von Übertragungseffekten im Innovationskontext und deren Auswirkungen auf die Wirtschaftlichkeit und das Verhalten von Unternehmen. Spillovers bezeichnen Wissensübetragungen von einer Firma auf eine oder mehrere andere Firmen. Diese Übertragungseffekte sind im Gegensatz zu unternehmerischen Allianzen unbeabsichtigt. Darüber hinaus beschreibt der Autor einen weiteren Effekt, den der \glqq technological opportunity\grqq{}. Die \glqq technological opportunity\grqq{} wird als exogene, technologisch bedingte, Schwankung in der Produktivität in Forschungs- und Entwicklungsarbeit definiert. \parencite{jaffe1986technological}. Es soll untersucht werden, ob eine Veränderung des Innovationsverhaltens einer Firma, nicht nur markt- bzw. nachfragebasiert ist, sondern zusätzlich von Änderungen in der technologische Landschaft abhängig ist.



\subsection{Bestimmung des Technologieraums}
\label{besttech}

Um \glqq Spillovers \grqq{} zu untersuchen müssen zunächst die technologischen Nachbarn von den zu betrachteten Firmen bestimmt werden. Im Allgemeinen soll also eine Darstellung gefunden werden, die den (technologische) Abstand von mehreren Unternehmen zueinander abbildet. Das Resultat dieser Darstellung wird als Technologieraum verstanden. In \parencite{jaffe1986technological} wird angenommen, dass die Patente von Firmen, die in verschiedenen Patentklassen verteilt sind, die Innovationsinteressen der jeweiligen Firma reflektiert. Hält also beispielsweise die Firma Honda viele Patente in dem Bereich \glqq Hybridfahrzeuge\grqq{}, so wird Honda auch zu einem großen Teil nach Technologien in dieser Richtung forschen. \\

Sei $f_{ik}$ der Anteil von Firma i's Patenten in Patentklasse k. Der Vektor $f_{i} = (f_{i1}...f_{iK})$ positioniert die Firma in einem $K-$dimensionalen Technologieraum \parencite{jaffe1989characterizing}. Es kann davon ausgegangen werden, dass zwei Firmen technologisch verwandt sind, wenn diese sich in dem Technologieraum nahe stehen. Um Ähnlichkeiten in Innovationsverhalten darzustellen, müssen die resultierenden Patentvektoren der Firmen in ein sinnvolles Verhältnis gesetzt werden. Dafür ist die Auswahl einer passenden Distanzmetrik ausschlaggebend. Eine Möglichkeit bestünde darin, zu schauen, in welche Richtung die Vektoren zeigen. In erster Linie soll das Verhältnis der Firmen zueinander bewertet werden, nicht wie viele Patente die Firmen jeweils insgesamt besitzen. Wir betrachten also nicht die Länge, sondern die Winkel der Vektoren zueinander. Eine passendes Maß dafür bietet die \glqq Cosine distance\grqq{}, oder auch Kosinus-Ähnlichkeit. 

\begin{equation}
\label{cosequ}
P_{ij}=\frac{ \sum_{k=1}^{K}{f_{ik} f_{jk}} }{ \sqrt{\sum_{k=1}^{K}{(f_{ik})^2}} \sqrt{\sum_{k=1}^{K}{(f_{jk})^2}} }
\end{equation}

Gewissermaßen misst $P_{ij}$ den Grad der Überschneidung der Vektoren $f_i$ und $f_j$. Je ähnlicher sich die Patente der Firmen i und j - im Bezug auf ihre Einteilung in die Patentklassen sind - desto größer wird der Zähler sein. Im Nenner wird das Ergebnis normalisiert. Sind i und j identisch, ist  $P_{ij} = 1$. Wie auch der Korrelationskoeffizienten ist diese Metrik symmetrisch, es gilt also $P_{ij} = P_{ji}$.  \\

In \parencite{jaffe1989characterizing} werden mit diesem Ansatz zunächst 10 verschiedene Unternehmen über 328 Patentklassen verglichen. Das Ergebnis bestätigt die Intuition. Alle Unternehmen in der Computerindustrie stehen sich nahe. Pharmaunternehmen sind weit entfernt von Unternehmen in den Bereichen Büro- und Schreibwaren, aber relativ nahe zu Firmen in Medizin- und Zahntechnik.  \\

\subsection{Clustering von Patentklassen}

Weiter soll gezeigt werden, wie die Patentklassen in Cluster zusammengefasst werden, deren Gruppierung sich am Markt beziehungsweise der Industrie orientieren. Zweck einer solchen Einteilung ist es die Interaktion zwischen Marktzweigen und Firmen einzufangen. Welche Folgen hat eine Veränderung eines Marktzweiges (Cluster), auf das strategische Verhalten einer Firma und vice-versa? \\

Das \glqq Clustering\grqq{} in \textcite{jaffe1989characterizing} ist ein iterativer Prozess, der sich an den \glqq $K-$means\grqq{} Algorithmus anlehnt. Dabei entspricht jeder Punkt im Raum des \glqq $K-$means\grqq{} der Verteilung der Patente einer Firma über alle Patentklassen. Die geometrische Mitte eines Clusters entspricht der durchschnittlichen Verteilung über alle Patentklassen, aller Firmen in unserem Industriecluster.
 
Für jede Patentklasse k, wird die Anzahl der Firmen ($C_k$) bestimmt, die Patente in dieser Kategorie halten. Jetzt wird \glqq ad-hoc \grqq{}, die Anzahl der $N$-höchsten $C_k$ bestimmt. Eine Firma wird dem Cluster zugeordnet, in dem sie am meisten Patente besitzt. Daraus ergeben sich die initialen Cluster\footnote{Bei $K-$means erfolgt die Initialisierung meistens zufällig}.
Für jedes Cluster wird die durchschnittliche Verteilung der Patente über alle Patentklassen berechnet (Analog: die Mittelpunkte der Cluster aus $K-$means). Dann wird die Verteilung der Patente über den Patentklassen für jede Firma einzeln berechnet (Analog: die einzelnen Punkte aus $K-$means). Passt die Verteilung eines Clusters, in der sich eine Firma nicht befindet besser\footnote{auf die Distanzmetrik zwischen Verteilungen wird nicht weiter eingegangen} auf die Verteilung der Firma selbst, so wechselt diese Firma in das Cluster. Dieser Prozess endet, wenn keine Firma mehr das Cluster wechselt. 
Zuletzt werden die Cluster je nach Zuordnung benannt, so ergeben sich beispielsweise die Cluster Chemie und Kohlenstoff, Nahrung, Medizin und Automobile. \\

Das \glqq Clustering\grqq{} wird für zwei Datensätze aus den Jahren 1972 und 1973 durchgeführt. Interessant sind diejenigen Firmen, die ihre Cluster zwischen den zwei Zeitperioden wechseln, oder anders: wir betrachten die Zu- und Abflüsse der Firmen innerhalb der verschiedenen Cluster. So kann beobachtet werden, dass \glqq generische\grqq{} Technologien, wie z.B. Beschichtungen, Fluid-Handling und Signalgebung, ihre \glqq Mitglieder\grqq{} aus vielen verschiedenen Industrien ziehen. Außerdem werden verschiedene interdisziplinäre Beziehungen zwischen Technologien festgestellt. Textile und Papier teilen Interessen mit den Industrien aus Beschichtungen und die Pharmaindustrie hängt stark mit Industrien für Medizintechnik und Apparaturen allgemein zusammen. Besonders auffallend sind die Ergebnisse wenn die Firmen einzeln betrachtet werden. So landen beispielsweise die drei großen Automobilhersteller alle in dem Cluster Triebwerke, die Automobilzulieferer hingegen, befinden sich in dem Automobilcluster. Xerox, ein Unternehmen für Bürobedarf, zentralisiert sich im Cluster für Elektrochemie. \\




\subsection{Übertragungseffekte und Technologische Opportunität}	
	
Mit dem definierten Technologieraum als Ausgangspunkt können nun die \glqq Spillovers\grqq{} untersucht werden. Demnach werden Firmen, die vielen anderen innovativen Firmen nahe stehen von deren Position profitieren können. Die potentiellen \glqq Spillover\grqq{} über dem Technologieraum sind nicht gleich verteilt, sie sind an den Punkten am höchsten, an denen sich die meisten Firmen befinden und somit am meisten potentielles Wissen an andere Firmen übertragen können. Alle potentiellen \glqq Spillovers\grqq{} werden als \glqq Spilloverpool\grqq{} definiert. Der \glqq Spilloverpool\grqq{} $S_i$ wird für jede Firma bestimmt und ergibt sich aus der gewichtet Summe der Innovationsarbeit aller anderen Firmen.

\begin{equation}
\label{spillequ}
S_i=\sum_{j \neq i} P_{ij} R_j
\end{equation}

Den Effekt den eigene Innovationen auf eine Firma hat ist in der realen Welt natürlich nicht direkt beobachtbar. In \textcite{jaffe1989characterizing}, werden dafür vier Indikatoren betrachtet: Patente, Bruttoeinnahmen, Kapitalertrag und Marktwert. Die Korrelation zwischen diesen Indikatoren und der innovatorischen Aktivität einer Firma sollte intuitiv klar sein. Ausgaben im Forschungs- und Entwicklungsbereich sind oft riskant, dennoch sollte die Findung neuer Technologien oder neuem Wissen früher oder später Einnahmen generieren und den Marktwert der Firma steigern. Mit Hilfe von Regressionstechniken wird versucht diese Indikatoren als Funktionen von Forschungsausgaben, Übertragungseffekten und weiteren Variablen zu approximieren. Beispielsweise ergibt sich die Patentfunktion aus einer modifizierte Cobb-Douglas Technologie \parencite{jaffe1986technological}. $k_i$ wird als das \glqq neu generierte Wissen\grqq{} einer Firma definiert.  

\begin{equation}
\label{patequ}
p_i= \beta_1 r_i + \beta_2 r_i s_i + \gamma_i s_i + \sum_{c = 1}^{21} (\delta_{1c} - \alpha_c) D_{ic} + \epsilon_{1i}
\end{equation}


$k_i$ wird als das \glqq neu generierte Wissen\grqq{} einer Firma definiert. $r_ i$ sind die Forschungsausgaben der Firma i und $s_i$ der Spilloverpool aus \ref{spillequ}. Die $D_{ic}$'s sind Dummyvariablen aus den oben angesprochenen Clustern, dabei wird angegeben ob eine Firma Patente in dem jeweiligen Cluster besitzt oder nicht. Ist die technologische Opportunität relevant so werden die $\delta_{1c}$'s, die Gewichtungen der Cluster, unterschiedlich ausfallen. So könnte man beispielsweise annehmen, dass das Cluster der Automobilindsturie eine höhere Gewichtung haben sollten, als das der Kühlindustrie.  $\epsilon_{1i}$ sind zufällige Störterme der einzelnen Firmen. Die Gleichung \ref{patequ} impliziert, dass die Forschungs- und Entwicklungsarbeit andere Firmen, den Wissensoutput einer Firma direkt erhöhen können.  Funktionen für Bruttoeinnahmen, Kapitalertrag und Marktwert werden analog definiert. Für weitere statistische Details siehe \textcite{jaffe1986technological}. \\


Nach Approximation der Koeffizienten kann beispielsweise berechnet werden, dass eine permanente Erhöhung der Forschungsausgaben um 10\%, eine durchschnittlich Patentsteigerung  um 8.8\% zufolge hat. Weiter steigen die Bruttoeinnahmen um 0.3\%, der Kapitalertrag um 1.8\% und der Marktwert um 3.6\%. Zusätzlich kann gezeigt werden, dass Firmen sich in die Cluster bewegen, die überdurchschnittlich hohe Einnahmen und Marktwerte aufweisen \parencite{jaffe1989characterizing}.  \\


Der in \ref{spillequ} definierte Spilloverpool spielt für die Erklärung aller vier Indikatoren eine wichtige Rolle. Zusätzlich gibt es eine Beziehung zwischen der firmeneigenen Forschungsarbeit und der Menge an erreichbaren Wissen für diese Firma. Die Produktivität der eigenen Innovationsarbeit hängt also von der Innovationsarbeit anderer, im Technologieraum naher, Firmen ab, welche wiederum von der eigenen Innovationsarbeit abhängt. Dieser Effekt führt dazu, dass der Spilloverpool für den Allgemeinfall nicht genau betrachtet werden kann. Die Firmen müssen zunächst nach ihrer eigenen innovativen Aktivität differenziert werden. \\

In \textcite{jaffe1986technological} wird dazu in drei Fällen unterschieden. Zunächst werden Firmen mit niedriger, dann mit durchschnittlicher und zuletzt in überdurchschnittlicher innovativen Aktivität betrachtet. Jetzt kann untersucht werden welchen Einfluss der Spilloverpool auf die Unternehmen hat. 
Die emittierten Patente und die Bruttoeinnahmen steigen mit den Übertragungseffekte für alle Firmen. Je größer der Spilloverpool, desto stärker ist der eigene technische Fortschritt. Ertrag und Marktwert sinken für wenig aktiven Firmen. Steigt die Aktivität der Firmen auf den Durchschnitt, so werden Ertrag und Marktwert positiv. Der Marktwert überdurchschnittlich aktiver Firmen steigt am stärksten\\

Nach diesen Ergebnissen wird der technologische Erfolg eines Nachbarn, es einer Firma also erleichtern selber technologische Fortschritte zu machen. Gleichzeitig wird dem wenig aktiven Unternehmen ein Markterfolg erschwert. Intuitiv ist diese Schlussfolgerung einleuchtend, denn technologische Nachbarn sind oft Konkurrenten. Ein schmaler technologischer Fortschritt wird für kurzfristige Mehreinnahmen sorgen, ein langfristiger Erfolg am Markt kann damit aber nicht garantiert werden. Auf der anderen Seite überwiegen die Vorteile der Übertragungseffekte für höchst innovative Firmen. Aufgrund des komplementären Effekts zwischen eigener Innovationsstärke und der, der technologischen Nachbarn, werden sich stark innovative Unternehmen nachhaltig am Markt behaupten können.

Auch die Gewichtungen $\delta_{1c}$ der Cluster sind für alle Performance Indikatoren (Beispiel: Patentindikator siehe \ref{patequ}) von großer Bedeutung. Hält man firmenspezifischen Attribute und Poolvariablen konstant, finden sich systematische Unterschiede in den Ergebnissen für verschieden Cluster. Zusätzlich sind die Cluster, die in den betrachteten Zeiträumen Firmen dazugewinnen, diese, die durchschnittlich höhere Profite und Marktwerte aufweisen. Das Modell deutet zumindest stichpunktartig auf die Nutzung innovativer Gelegenheiten hin, ob es sich dabei allerdings um einen \glqq market-pull\grqq{} oder \glqq technology push\grqq{} oder möglicherweise um eine Kombination beider Effekte handelt, lässt der Autor weitestgehend offen.



\section{Der Technologieraum für die Firma Honda und ihre Konkurrenz nach \textcite{jaffe1986technological}}
\subsection{Firmenauswahl}

Als Datengrundlage nutzen wir die in Kapitel 4 angesprochenen Patente aus der CPC-Klasse $Y02T\_10$. Der Schwerpunkt unserer Analyse wird also stets vor dem Hintergrund sein, Firmen bzw. Patente im Bereich von innovativen, klimaneutrale Technologien in der Automobilbranche zu untersuchen. Um einen Technologieraum nach \textcite{jaffe1986technological} für die Firma Honda zu erstellen, bestimmen wir zunächst alle Firmen, die wir mit Honda in ein Verhältnis setzen wollen. In \textcite{jaffe1989characterizing}, werden \glqq willkürlich\grqq{} zehn Firmen ausgewählt, die sich in ihrer Position am Markt teilweise stark unterscheiden. Wir wollen versuchen die Position der Firma Honda und ihrer Konkurrenten möglichst realistisch darzustellen, dafür suchen wir uns zunächst die aktivsten Unternehmen in diesem Sektor. Aktivität definieren wir dabei als die Anzahl der Patente, die eine Firma in der Klasse $Y02T\_10$ besitzt. Bei der Anzahl der Firmen halten wir uns an \textcite{jaffe1989characterizing}, wir wollen einen aussagekräftigen aber gleichzeitig übersichtlichen Technologieraum. In \textcite{benner2008close} wird gezeigt, dass eine geringe Datenmenge zu sehr unpräzisen, teilweise widersprüchlichen, Ergebnissen führen kann.
Mit dieser Auswahl kann also außerdem garantiert werden, dass unser \glqq patent bias\grqq{}, möglichst gering gehalten wird. \\

\begin{figure}[ht]
\centering
\includegraphics[width=1.0\linewidth]{files/top10.PNG}
\caption{Patentzahl Top 10 Unternehmen}
\label{top10}
\end{figure}

In \ref{top10}, werden die zehn aktivsten Firmen mit ihren jeweiligen Patentzahlen abgebildet. Wir haben insgesamt sechs japanische Unternehmen (Toyota, Nissan, Honda, Denso, Mazda und Hitachi), zwei Unternehmen mit amerikanischem Ursprung (Ford und GM), ein deutsches Unternehmen (Bosch) und Hyundai aus Südkorea. Bosch und Denso sind Automobilzulieferer, die restlichen Unternehmen sind Automobilhersteller im klassischen Sinne. Mit über 30000 angemeldeten Patenten, fällt der japanische Automobilhersteller Toyota klar aus dem Muster. Die neun anderen Firmen befinden sich alle in einem Intervall von $(4000, 10000)$ Patenten. Honda liegt mit knapp über 7000 angemeldeten Patenten auf dem vierten Platz hinter Toyota, Nissan und Bosch. \\

Patente, die von verschiedenen Unternehmen angemeldet wurden, dennoch demselben Konzern angehören dürfen nicht vergessen werden. So haben wir beispielsweise die Unternehmen \glqq Toyota Motor Corporation\grqq{}, \glqq Toyota Industries Corporation\grqq{} und \glqq Toyota Central Research \& Development Lab\grqq{} zusammengefasst.


\subsection{Bildung der Patentvektoren und der Distanzmatrix}

Wir wollen die in \ref{besttech} vorgestellten Vektoren $f_{ik}$ für unsere Firmen und unsere Patentklassen bilden. Wir betrachten zunächst die ersten vier Stellen der CPC-Klassen und fassen alle Patente nach ihrer Gruppe zusammen. 

\begin{figure}[ht]
\centering
\includegraphics[width=1.0\linewidth]{files/cpcnum15.PNG}
\caption{Anzahl der $Y02T\_10$ Patente nach Gruppierung}
\label{cpcnum15}
\end{figure}

Für die Vektorlänge wählen wir im ersten Schritt ad-hoc $k=15$. In \ref{cpcnum15} sehen wir die Verteilung aller $Y02T\_10$ Patente auf die fünfzehn häufigsten CPC-Klassen. Bei der Gruppe F02D handelt es sich um Patente im Bereich \glqq Controlling Combustion Engines\grqq{}. Diese Klasse beinhaltet, für die Automobilindustrie eher klassische Pantente wie \glqq Treibstoffkontrollsysteme \grqq{} und \glqq Motorkontrollsysteme \grqq{}. Interessant ist beispielsweise die Klasse B60L. Dabei handelt es sich um Patente im Bereich der Elektroantriebe. So geht es hier um Themen wie \glqq  Electric propulsion with power supply from forces of nature, e.g. sun or wind\grqq{} und \glqq Methods, circuits, or devices for controlling the traction-motor speed of electrically-propelled vehicles\grqq{}. Fraglich ist ob wir unsere Patente wie in \textcite{jaffe1989characterizing} clustern sollten. Würde man alle Patente in der Automobilindustrie betrachten, wäre eine Einteilung in einzelne Bereiche sicher sinvoll. Da wir aber explizit Patente einer Gruppe betrachten ($Y02T\_10$), ist ein Sinnzusammenhang inhärent gegeben. Ein Clustering ist für unseren Daten also überflüssig.

Für die Einteilung der Patentvektoren folgen wir dem Vorgehen von \textcite{jaffe1989characterizing} und ordnen jeder Firma einen Vektor mit den in \ref{cpcnum15} abgebildeten CPC-Klassen zu. Wir gruppieren unsere Daten also nach Firmen und zählen die jeweils angemeldeten Patente pro Patentklasse. Wir erhalten also beispielsweise $f_{honda \; F02P} = 371$ und $f_{honda F02D} = 4710$. So ergibt sich für die Firma Honda folgender Vektor:

\begin{equation*}
f_{honda} = \{371, 773, 628, 496, 478, 520, 627, 902, 1690, 2660, 2003, 2349, 3522, 7218, 4710\}
\end{equation*}

Wir bilden diesen Vektor für jede Firma und wenden anschließend die Metrik aus \ref{cosequ} an. Es ergibt sich folgende Distanzmatrix: \\

\vphantom{dasdddddddddddddddddddddddddddddddddddddddddddddddddddddddddddddddddddddddddddddddddddddddddddddddddddddddddddddddddddd}

%\NiceMatrixOptions{code-for-first-col = \color{blue}, code-for-last-row = \color{blue}}

$\begin{pNiceMatrix}[first-col, last-row]% don't forget the %
\text{toyota } \; \; \;    & 1 &  \\
\text{nissan   } \; \; \;   & 0.963 & 1 &  \\
\text{bosch   }  \; \; \; & 0.913 &  0.901 & 1 & \\
\text{honda   } \; \; \;   & 0.982 & 0.968 & 0.891 & 1 &  \\
\text{denso   } \; \; \;   & 0.910 & 0.926 & 0.972 & 0.898 & 1 &  \\
\text{hyundai   } \; \; \; & 0.950 & 0.908 & 0.885 & 0.953 & 0.879 & 1 &  \\
\text{ford   } \; \; \;     & 0.962 & 0.957 & 0.940 & 0.948 & 0.952 & 0.951 & 1 &  \\
\text{mazda   } \; \; \;  & 0.656 & 0.816 & 0.656 & 0.686 & 0.700 & 0.605 & 0.741 & 1 &  \\
\text{hitachi   }\; \; \;  & 0.891 & 0.913 & 0.844 & 0.923 & 0.869 & 0.826 & 0.827 & 0.626 & 1 &  \\
\text{gm   } \; \; \;        & 0.974 & 0.902 & 0.883 & 0.948 & 0.855 & 0.936 & 0.929 & 0.555 & 0.814 & 1 \\
                          &\text{toyota} & \text{nissan} & \text{bosch} & \text{honda} & \text{denso} & \text{hyundai} & \text{ford} & \text{mazda} & \text{hitachi} & \text{gm}
                      
\end{pNiceMatrix}$




\vphantom{dasdddddddddddddddddddddddddddddddddddddddddddddddddddddddddddddddddddddddddddddddddddddddddddddddddddddddddddddddddddd}

Wir lesen: $P_{honda \; toyota} = P_{toyota \; honda} = 0.982$. Wie bereits erwähnt wird der Zähler bei der Kosinus-Ähnlichkeit im Nenner normalisiert. Die Distanzen befinden sich in einem Wertebereich von 0 bis 1, wobei die Firmen bei 1 identisch sind und bei 0 keine Patente in derselben CPC-Klasse halten. Betrachtet man die einzelnen Werte der Matrix, fällt auf, dass sich alle Firmen, bis auf Mazda, sehr nahe stehen. Intuitiv ist dieses Ergebnis sinnvoll. Anders als in \textcite{jaffe1989characterizing} vergleichen wir ausschließlich Unternehmen aus einem Industriezweig. Zusätzlich sind die von uns gewählten CPC-Klassen relativ ungenau. Wir betrachten lediglich die CPC-Klassen die uns durch die Gesamtpatentanzahl vorgegeben wird, es ist also möglich, dass eine Firma Patente in einer CPC-Klasse hält, diese aber nicht in unserem Vektor vorkommt. Durch diesen Effekt \glqq verlieren\grqq{} wir potentielle Patentklassen, welche unsere Firmen weiter unterscheiden könnten.  Um das Problem zu umgehen wählen wir eine andere Vektorzusammensetzung: Sei $M$ die Menge der von uns betrachteten CPC-Klassen. Wähle die größten 30 CPC-Klassen $c_{ik}$ $(i = 1 ... 30)$ für jede unserer zehn Firmen k. Ist $c_{ik}$ nicht in M. Füge $c_{ik}$ $M$ hinzu. \\

Um für bessere Interpretierbarkeit zu sorgen, wollen wir im nächsten Abschnitt die resultierende Distanzmatrix mittels multidimensionaler Skalierung in einer zweidimensionalen Ebene darstellen.

\subsection{Multidimensionale Skalierung}
\label{mds}

Multidimensionale Skalierung (MDS) ist ein Berechnungsverfahren, das dabei helfen soll verdeckte Strukturen in Daten aufzuzeigen \parencite{kruskal1978multidimensional}. Angenommen man hat eine Karte, die verschiedenen Position amerikanischer Städte zeigt. Durch einfaches messen der Abstände zwischen den Städten, lässt sich eine Tabelle mit den jeweiligen Abstandswerten ausfüllen. Die Rückrichtung dieser Aufgabe ist das Problem der MDS \parencite{kruskal1978multidimensional}.
Im Prinzip geht es also darum N Objekte, geometrisch durch N Punkte zu repräsentieren. Dabei sollen die Abstände zwischen den Punkte im zweidimensionalen Raum, die Unterschiede der Objekte (hier: unsere zehn Firmen) reflektieren können. Bei der klassischen MDS werden die Distanzen zwischen den Objekten mit der euklidischen Metrik berechnet. Da wir bereits eine \glqq eigene\grqq{} Distanzmetrik haben, fällt dieser Schritt weg. Ausgehend von unserer Matrix folgen alle weiteren Schritte der MDS dem Shepard-Kruskal Algorithmus. Das Ergebnis der multidimensionalen Skalierung ist, bis auf Rotation und Skalierung, eindeutig. Umgesetzt wird die multidimensionale Skalierung mithilfe der freien Software-Bibliothek \glqq Scikit-learn\grqq{}. 

\vphantom{dasdddddddddddddddddddddddddddddddddddddddddddddddddddddddddddddddddddddddddddddddddddddddddddddddddddddddddddddddddddd}

\begin{lstlisting}[style=python]
# Importieren der Bibliothek
from sklearn.manifold import MDS 

# Setze Anzahl der CPC-Klassen pro Firma und random_state für Reproduzierbarkeit:
def mds(anzahl=30, random_sate = 42):
    
    # Berechnung der Distanzen, Metrik ist die Kosinus-Ähnlichkeit
    jt = jaffetable(cosine_distance, anzahl)
    d = jt.to_numpy()
    N = d.shape[0]
    
    # Korrektur der Indizes
    for i in range(N): 
        for j in range(i+1,N):
            d[i,j] = d[j,i]
            
    # Initalisierung der MDS, zwei Dimensionen, Distanzmatrix ist vorgegeben
    embedding = MDS(n_components=2, dissimilarity = 'precomputed', random_state = random_state)
    
    # Durchführung der MDS
    m = embedding.fit_transform(d) 
    
    # Erstellung des Plots
    fig = plt.figure(figsize=(8,8)) 
    ax = fig.add_subplot()
    plt.axis('equal')

    # Übertragung der Koordinanten in den Plot 
    plt.scatter(m[:, 0], m[:, 1], color='navy')
    
    # Beschriftung der Punkte
    for i in range(N):
        plt.annotate(jt.index[i].lower(),(m[i,0] + 0.01, m[i,1]))
\end{lstlisting}

\vphantom{dasdddddddddddddddddddddddddddddddddddddddddddddddddddddddddddddddddddddddddddddddddddddddddddddddddddddddddddddddddddd}

\begin{figure}[!ht]
\centering
\includegraphics[width=1.0\linewidth]{files/mdsjaffe1.PNG}
\caption{Graph der multidimensionalen Skalierung nach Patentvektoren}
\label{mdsjaffe1}
\end{figure}


\subsection{Interpretation des Graphen}

Bei der Interpretation eines MDS-Graphen ist es immer wichtig zu beachten, dass sowohl die Achsen, als auch die absoluten Positionen der Firmen keine Rolle spielen. Es geht lediglich um das Verhältnis der Firmen zueinander. Würde man beispielsweise eine weitere Firma in die Berechnung aufnehmen, werden alle Firmen ihre Positionen ändern, je nachdem in welchem Verhältnis diese zu der neuen, und den anderen, Firmen stehen. \\

Wir erkennen zunächst ein kleineres Firmencluster, bestehend aus Ford, Bosch und Denso. Da Bosch und Denso Automobilzulieferer sind, könnte man davon ausgehen, dass beide Unternehmen in ähnlichen technologische Sektoren aktiv sind. So stellen Automobilzulieferer oft Einzelteile wie z.B. Zündkerzen und Schrauben her. Das Cluster Ford, Bosch, Denso (und GM) unterscheidet sich primär durch starke Aktivität in der Patentklassen\footnote{Wir betrachten die Patentklassen in Relation zu der jeweilige Gesamtpatentanzahl} F02D (Verbrennungsmotor) und vergleichsweise schwache Aktivitäten in der Patentklasse mit Hauptfokus auf Elektromobilität (B60L). Als zweite Gruppe beobachten wir die drei größten japanischen Automobilhersteller: Toyota, Honda und Nissan. Mit 32\% aller Neuzulassungen in Japan liegt Toyota an der Spitze, gefolgt von Honda und Nissan zu je 15\%.\footnote{Quelle: https://www.autoscout24.de/auto/japanische-automarken/ Stand: 6.10.2020}. Die Graphik legt nahe, dass das Patentierverhalten der drei Firmen ähnlich sein muss. Tatsächlich sind die Unternehmen, bis auf kleinere Abweichungen, in denselben Patentklassen aktiv. Primär beobachten wir überdurchschnittliche Aktivitäten in der CPC-Klassen B60L (Elektromobilität). Am interessantesten jedoch ist die Position von Mazda. Während alle anderen Unternehmen mehr oder weniger nahe zueinander stehen, fällt Mazda gänzlich aus dem Muster. Grund dafür ist eine beachtliche Patentzahl in der CPC-Klasse F02D. Genauer handelt es dabei um die Patente der Klassen F02B53/00 und F02B2053/005. Die zugrundeliegende Technologie dieser Patente ist der Wankelmotor. "Mazda hat in Japan einen außergewöhnlichen Hybrid-Antrieb mit Wankelmotor patentieren lassen. Das System arbeitet mit ingesamt drei E-Motoren und soll laut Mazda vor allem leichter sein als aktuelle elektrische Allradantriebe."\footnote{Quelle: https://www.electrive.net/2020/04/20/ Stand: 06.10.2020}. Um Aussagen über die Entwicklung der Technologien zu machen und mögliche Zielrichtung der Unternehmen festzustellen wäre es interessant zu sehen, wie sich die Position der Firmen über mehrere Zeiträume hinweg verändern.


\subsection{Zeitliche Entwicklung}

In \textcite{jaffe1986technological} wird untersucht welche Technologiecluster über die Zeit an Mitglieder \glqq gewinnen\grqq{}. Wir wollen zeigen in welche Richtungen sich die Firmen in unserem MDS Plot über zwei Zeiträume hinweg bewegen. Dabei ergibt sich jedoch folgendes Problem: Die Abstände der Firmen zueinander ist immer eindeutig, allerdings können die Firmen im resultierende MDS Graph rotieren. Für dieselbe Lösung gibt es also eine Vielzahl unterschiedlicher Graphen. Das macht es für uns zunächst unmöglich zwei Graphen miteinander zu vergleichen, beziehungsweise die Bewegungsrichtung einer Firma aufzuzeigen. Mithilfe des setzen einer \glqq random\_state\grqq{}-Variable, lassen sich zwar Graphen genau reproduzieren, allerdings verändern sich unseren Ausgangsdaten für den zweiten Zeitraum, damit besteht das Problem also weiterhin. Wir schlagen folgende Lösung vor: wir berechnen den Graphen für den ersten Zeitraum und merken uns die Koordinaten der Punkte. Für den zweiten Graphen berechnen wir den Abstand der resultierenden Koordinaten zu den Koordinaten des ersten Graphen für (beispielsweise) 10000 verschiedene \glqq random\_states\grqq{}. Wir verwenden schließlich die Koordinaten, die den geringsten Abstand zu den Koordinaten des ersten Graphen aufweisen. Damit garantieren wir, dass sich die beiden Graphen sinnvoll vergleichen lassen, ohne dabei falsch implizierte Bewegung der Firmen, durch Rotation des Graphen aufgrund der MDS zu erhalten.\\

Wir betrachten die Zeiträume der Jahre 2000 bis 2010 und 2011 bis 2018. Daraus ergeben sich die Graphen \ref{plot2}

\begin{itemize}

         \item Plot
         
\end{itemize}

\subsection{Robustheit}

Um die Robustheit der Ergebnisse zu garantieren wurden folgende Maßnahmen durchgeführt. Naheliegend ist es zunächst die Länge der Patentvektoren (K) zu variieren. Je nach Vektorlänge werden mehr oder weniger Patentklassen für die Berechnung berücksichtigt. Um alle relevanten Patentklassen in die Berechnung aufzunehmen, sollte K also möglichst hoch gehalten werden, dabei ist das Ergebnis bei einem Wert von $K=30$, schon sehr robust, eine weitere Erhöhung der Vektorlänge führt zu keiner signifikanten Veränderung im Graphen. Als nächstes wurde die Granularität der betrachten Patentklassen verändert. Bis jetzt haben wir nur die ersten vier Stellen der CPC-Klassen betrachtet, davon ausgehend wurden die Patentzahlen der Firmen gruppiert. Betrachtet man beispielsweise die ersten sieben Stellen der CPC-Klassen werden die Patentvektoren \glqq feiner\grqq{}. So wird die CPC-Klasse F02D, beispielsweise zu F02D\_13 und F02D\_41. Auch hier finden wir keine signifikante Veränderung des Graphen. Da die Wahl der Distanzmetrik Ergebnisse stark beeinflussen kann, wurde im letzten Schritt statt der Kosinus-Ähnlichkeit die in \textcite{bar2012measure} beschriebene \glqq min-complement distance\grqq{}. 

\begin{equation}
\label{mincequ}
P_ij= 1 - \sum_{k = 1}^K \min\{p_{ik}, p_{jk}\}
\end{equation}

Die Motivation hinter der wohldefinierten Metrik ist es irrelevante Patentklassen für die Berechnung der Distanzen zwischen den Firmen i und j nicht zu berücksichtigen. Irrelevante Patentklassen gelten als solche, wenn eine Firma in dieser Patentklasse keine Patente hält. \textcite{bar2012measure} behauptet, dass die technologische Distanz zwischen zwei Firmen, von den Patentklassen abhängen sollte, in denen beide Firmen aktiv sind. In Gleichung \ref{mincequ} werden jeweils die Minima der Patentzahlen pro Klasse und Firma betrachtet. Ergibt sich für den Term $\min\{p_{ik}, p_{jk}\} = 0$, hat das keinen Einfluss auf die Summe. Im Kontext der Robustheitsprüfung ergab das Anwenden der \glqq min-complement distance\grqq{}, die größte Veränderung (siehe \ref{GRAFIKzwei}). Allerdings bleiben die oben angesprochenen Gruppierungen der Firmen weiterhin bestehen. Letztendlich lässt sich also sagen, dass der Graph der MDS und unser Technologieraum nach \textcite{jaffe1986technological} in seinem Ergebnis robust ist.

\subsection{Zusammenfassung}
